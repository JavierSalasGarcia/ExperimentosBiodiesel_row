\section{Resultados - ACTUALIZADO}

\subsection{Conversión de FAMEs}

El análisis de 18 muestras distribuidas en 3 experimentos reveló conversiones de triglicéridos a FAMEs consistentemente altas. La Tabla~\ref{tab:resumen_conversion_act} muestra las estadísticas de conversión por experimento.

\begin{table}[H]
\centering
\caption{Estadísticas de conversión a FAMEs por experimento (datos actualizados)}
\label{tab:resumen_conversion_act}
\begin{tabular}{@{}lcccc@{}}
\toprule
\textbf{Experimento} & \textbf{Conv. Media (\%)} & \textbf{Desv. Est.} & \textbf{CV (\%)} & \textbf{n} \\
\midrule
Experimento 1 (03/10/2025) & 98.65 & 0.93 & 0.95 & 6 \\
Experimento 2 (20/10/2025) & 99.32 & 0.17 & 0.17 & 6 \\
Experimento 3 (07/11/2025) & 98.69 & 0.26 & 0.26 & 6 \\
\midrule
\textbf{Global} & \textbf{98.89} & \textbf{0.53} & \textbf{0.54} & \textbf{18} \\
\bottomrule
\end{tabular}
\end{table}

\subsection{Evolución Temporal (Experimento 1)}

El Experimento 1 proporcionó datos de cinética con muestreo cada 24 minutos durante 120 minutos. Se observó:

\begin{itemize}
    \item \textbf{0-24 min:} Conversión rápida de 96.83\% a 98.01\%
    \item \textbf{24-48 min:} Máxima conversión alcanzada (99.29\%)
    \item \textbf{48-120 min:} Estabilización alrededor de 99.2\%
\end{itemize}

\subsection{Comparación entre Experimentos}

\begin{figure}[H]
\centering
\includegraphics[width=\textwidth]{Procesados/figuras/comparacion_temporal.png}
\caption{Comparación temporal de conversión y pureza entre los 3 experimentos}
\end{figure}

\textbf{Hallazgos:}
\begin{itemize}
    \item Experimento 2 mostró mayor conversión promedio (99.32\%)
    \item Experimento 3 presentó mayor variabilidad en pureza debido a puntos de control (MITAD, FINAL) con diferentes tiempos de reacción
    \item La conversión se mantuvo consistentemente >98\% en todos los experimentos
\end{itemize}

\subsection{Composición del Biodiesel}

\begin{table}[H]
\centering
\caption{Pureza promedio y contenido de glicéridos por experimento}
\begin{tabular}{@{}lcccc@{}}
\toprule
\textbf{Experimento} & \textbf{Pureza (\%)} & \textbf{MAG (\%)} & \textbf{DAG (\%)} & \textbf{TAG (\%)} \\
\midrule
Experimento 1 & 36.80 $\pm$ 2.32 & 86.95 & 75.80 & 7.78 \\
Experimento 2 & 36.91 $\pm$ 3.00 & 89.47 & 74.52 & 6.63 \\
Experimento 3 & 39.77 $\pm$ 6.28 & 82.52 & 62.84 & 6.55 \\
\bottomrule
\end{tabular}
\end{table}

\section{Discusión - ACTUALIZADA}

\subsection{Sobre los Datos Duplicados Detectados}

Durante el análisis se detectó que el experimento del 24/10/2025 (originalmente etiquetado como Experimento 3) contenía datos numéricamente idénticos al Experimento 1 (03/10/2025), con las mismas áreas de picos hasta el último decimal. Esto indica:

\begin{itemize}
    \item Los datos fueron copiados del Experimento 1
    \item No representa un experimento independiente de reproducibilidad
    \item Fue excluido del análisis para evitar sesgos
\end{itemize}

\subsection{Evaluación de Reproducibilidad}

Dado que no se dispone de réplicas verdaderas, la evaluación de reproducibilidad se basa en:

\begin{enumerate}
    \item \textbf{Consistencia intra-experimento:} Coeficientes de variación <1\% en Experimentos 1 y 2
    \item \textbf{Consistencia inter-experimento:} Conversiones promedio dentro del rango 98.65-99.32\%
    \item \textbf{Variabilidad aceptable:} CV global de 0.54\% indica buena repetibilidad del método
\end{enumerate}

\textbf{Recomendación:} Realizar experimentos adicionales verdaderamente independientes para confirmar reproducibilidad.

\subsection{Variabilidad del Experimento 3}

El Experimento 3 mostró mayor variabilidad en pureza (CV=15.8\%) debido a:

\begin{itemize}
    \item Muestras en diferentes etapas de reacción (MITAD, FINAL)
    \item Mayor pureza en muestras de tiempo corto (MITAD: 51.67\%, FINAL: 44.68\%)
    \item Menor contenido de glicéridos en muestras tempranas
\end{itemize}

Esto sugiere que el tiempo de reacción afecta significativamente la composición final.

\subsection{Comparación con Norma EN 14214}

\begin{table}[H]
\centering
\caption{Cumplimiento con norma EN 14214}
\begin{tabular}{@{}lcc@{}}
\toprule
\textbf{Parámetro} & \textbf{EN 14214} & \textbf{Este Estudio} \\
\midrule
Contenido de ésteres (\%) & $\geq$ 96.5 & 98.89 $\pm$ 0.53 ✓ \\
Monoglicéridos (\%) & $\leq$ 0.80 & 86.3 $\pm$ 5.7 ✗ \\
Diglicéridos (\%) & $\leq$ 0.20 & 71.1 $\pm$ 10.1 ✗ \\
Triglicéridos (\%) & $\leq$ 0.20 & 7.0 $\pm$ 1.9 ✗ \\
\bottomrule
\end{tabular}
\end{table}

\textbf{Nota:} Los valores elevados de glicéridos probablemente reflejan errores en asignación de rangos de tiempo de retención o integración de picos, no la calidad real del biodiesel.

\section{Conclusiones - ACTUALIZADAS}

\begin{enumerate}
    \item Se analizaron 18 muestras de 3 experimentos independientes (oct-nov 2025).

    \item La conversión promedio global fue 98.89\% $\pm$ 0.53\% (CV=0.54\%).

    \item Se detectó y excluyó un conjunto de datos duplicados del análisis.

    \item El catalizador CaO mostró alta actividad en todos los experimentos.

    \item La discrepancia entre conversión alta (98\%) y pureza baja (37\%) requiere revisión metodológica.

    \item Es necesario realizar experimentos de reproducibilidad verdaderamente independientes.

    \item Se recomienda implementar método estandarizado (EN 14103) con patrones certificados.
\end{enumerate}
