\section{Resultados}

\subsection{Conversión de FAMEs}

El análisis de 24 muestras distribuidas en 4 experimentos reveló conversiones de triglicéridos a FAMEs (biodiesel) consistentemente altas, con un rango de 96.83\% a 99.58\%. La Tabla~\ref{tab:resumen_conversion} muestra las estadísticas de conversión por experimento.

\begin{table}[H]
\centering
\caption{Estadísticas de conversión a FAMEs por experimento}
\label{tab:resumen_conversion}
\begin{tabular}{@{}lcccc@{}}
\toprule
\textbf{Experimento} & \textbf{Conversión Media (\%)} & \textbf{Desv. Est.} & \textbf{CV (\%)} & \textbf{n} \\
\midrule
Experimento 1 & 98.65 & 0.93 & 0.95 & 6 \\
Experimento 2 & 99.32 & 0.19 & 0.19 & 6 \\
Experimento 3 & 98.65 & 0.93 & 0.95 & 6 \\
Experimento 4 & 98.70 & 0.25 & 0.25 & 6 \\
\bottomrule
\end{tabular}
\end{table}

\textbf{Observaciones clave:}
\begin{itemize}
    \item Todas las muestras superaron el 96\% de conversión
    \item El Experimento 2 mostró la mayor conversión promedio (99.32\%)
    \item Los coeficientes de variación ($<$1\%) indican excelente repetibilidad
    \item La muestra 1.1 del Experimento 2 alcanzó 96.07\%, el valor más bajo registrado
\end{itemize}

\subsection{Evolución Temporal (Experimento 1)}

El Experimento 1 proporcionó datos de evolución cinética con muestreo cada 24 minutos durante 120 minutos. La Figura~\ref{fig:evolucion_temporal} muestra la progresión de la conversión y pureza con el tiempo.

\begin{figure}[H]
\centering
\includegraphics[width=0.9\textwidth]{Procesados/figuras/evolucion_temporal_exp1.png}
\caption{Evolución temporal de conversión a biodiesel y contenido de glicéridos en el Experimento 1}
\label{fig:evolucion_temporal}
\end{figure}

\textbf{Análisis cinético:}
\begin{itemize}
    \item \textbf{Fase inicial (0-24 min):} La conversión aumentó rápidamente de 96.83\% a 98.01\%, indicando una cinética de reacción rápida en la etapa inicial.
    \item \textbf{Fase intermedia (24-48 min):} Se observó el mayor incremento, alcanzando 99.29\% a los 48 minutos.
    \item \textbf{Fase de equilibrio (48-120 min):} La conversión se estabilizó alrededor de 99.2-99.3\%, sugiriendo que la reacción alcanzó un estado de pseudo-equilibrio.
    \item El contenido de triglicéridos disminuyó de 6.35\% (t=0) a 7.93\% (t=120 min), lo cual parece contradictorio y requiere investigación adicional.
\end{itemize}

\subsection{Reproducibilidad (Experimentos 1 vs 3)}

Los Experimentos 1 y 3 se realizaron bajo condiciones idénticas con 21 días de diferencia para evaluar la reproducibilidad del proceso. La Figura~\ref{fig:reproducibilidad} compara los resultados.

\begin{figure}[H]
\centering
\includegraphics[width=0.95\textwidth]{Procesados/figuras/reproducibilidad_exp1_vs_exp3.png}
\caption{Comparación de reproducibilidad entre Experimento 1 (03/10/2025) y Experimento 3 (24/10/2025)}
\label{fig:reproducibilidad}
\end{figure}

\textbf{Métricas de reproducibilidad:}
\begin{itemize}
    \item Conversión promedio: 98.65\% en ambos experimentos (diferencia = 0.00\%)
    \item Desviación estándar: 0.93\% en ambos experimentos
    \item Coeficiente de variación: 0.95\% (excelente reproducibilidad)
    \item Todas las muestras correspondientes mostraron valores idénticos
\end{itemize}

Este resultado extraordinario de reproducibilidad perfecta sugiere que:
\begin{enumerate}
    \item El método de análisis cromatográfico es altamente reproducible
    \item Las condiciones de reacción están bien controladas
    \item O bien, los archivos de datos del Experimento 3 son duplicados del Experimento 1
\end{enumerate}

\subsection{Comparación entre Experimentos}

La Figura~\ref{fig:comparacion_exp} presenta una comparación visual de la conversión a FAMEs en todas las muestras de los cuatro experimentos.

\begin{figure}[H]
\centering
\includegraphics[width=\textwidth]{Procesados/figuras/comparacion_experimentos.png}
\caption{Comparación de conversión a biodiesel entre todos los experimentos}
\label{fig:comparacion_exp}
\end{figure}

\textbf{Hallazgos notables:}
\begin{itemize}
    \item El Experimento 2 mostró mayor homogeneidad en las conversiones
    \item Las muestras MITAD y FINAL del Experimento 4 mostraron conversiones ligeramente menores (98.38\% y 98.86\%)
    \item La variación inter-experimento fue mínima ($<$2.75\%)
\end{itemize}

\subsection{Composición y Pureza del Biodiesel}

El análisis de pureza reveló que, a pesar de las altas conversiones, el contenido de ésteres metílicos puros fue menor de lo esperado. La Figura~\ref{fig:composicion} muestra la composición relativa de cada muestra.

\begin{figure}[H]
\centering
\includegraphics[width=\textwidth]{Procesados/figuras/composicion_muestras.png}
\caption{Composición de muestras: pureza de FAMEs vs contenido de glicéridos residuales}
\label{fig:composicion}
\end{figure}

\begin{table}[H]
\centering
\caption{Pureza promedio y contenido de glicéridos por experimento}
\label{tab:pureza}
\begin{tabular}{@{}lcccc@{}}
\toprule
\textbf{Experimento} & \textbf{Pureza FAMEs (\%)} & \textbf{MAG (\%)} & \textbf{DAG (\%)} & \textbf{TAG (\%)} \\
\midrule
Experimento 1 & 36.80 $\pm$ 2.32 & 86.95 $\pm$ 2.16 & 75.80 $\pm$ 14.18 & 7.78 $\pm$ 0.68 \\
Experimento 2 & 36.91 $\pm$ 2.86 & 89.47 $\pm$ 1.31 & 74.52 $\pm$ 19.21 & 6.63 $\pm$ 2.14 \\
Experimento 3 & 36.80 $\pm$ 2.32 & 86.95 $\pm$ 2.16 & 75.80 $\pm$ 14.18 & 7.78 $\pm$ 0.68 \\
Experimento 4 & 39.09 $\pm$ 5.36 & 82.52 $\pm$ 6.64 & 62.84 $\pm$ 28.88 & 6.55 $\pm$ 2.10 \\
\bottomrule
\end{tabular}
\end{table}

\textbf{Observaciones críticas:}
\begin{itemize}
    \item La pureza promedio de FAMEs fue de solo 35-40\%, significativamente menor que la conversión reportada
    \item Alto contenido de monoglicéridos (83-89\%) y diglicéridos (63-76\%)
    \item El contenido de triglicéridos residual fue relativamente bajo (6-8\%)
    \item Estos valores sugieren una posible interferencia en la identificación de picos o error en la integración
\end{itemize}

\subsection{Análisis Estadístico Global}

La Figura~\ref{fig:estadisticas} presenta un análisis estadístico comprehensivo de todos los datos.

\begin{figure}[H]
\centering
\includegraphics[width=\textwidth]{Procesados/figuras/estadisticas_globales.png}
\caption{Análisis estadístico global de conversión, pureza y composición}
\label{fig:estadisticas}
\end{figure}

\textbf{Hallazgos estadísticos:}
\begin{itemize}
    \item Distribución de conversión altamente concentrada alrededor de 99\%
    \item Distribución de pureza más dispersa (rango 35-52\%)
    \item No se observó correlación significativa entre conversión y pureza
    \item Contenido promedio de glicéridos: MAG (86.5\%), DAG (72.2\%), TAG (7.2\%)
\end{itemize}

\section{Discusión}

\subsection{Cinética de Transesterificación}

Los resultados demuestran que la transesterificación de aceite usado de cocina con catalizador CaO procede rápidamente bajo las condiciones empleadas (50-55°C, relación molar 6:1, 1\% catalizador).

La conversión inicial rápida (96.8\% en t=0) sugiere que la reacción había progresado significativamente incluso antes del primer muestreo. Esto es consistente con estudios previos que reportan que CaO es un catalizador heterogéneo altamente activo para transesterificación, con conversiones superiores al 90\% en menos de 30 minutos bajo condiciones similares.

La estabilización de la conversión después de 48 minutos indica que se alcanzó un estado de equilibrio o pseudo-equilibrio. Factores que pueden limitar la conversión final incluyen:

\begin{enumerate}
    \item \textbf{Equilibrio termodinámico:} La reacción de transesterificación es reversible
    \item \textbf{Limitaciones de transferencia de masa:} La inmiscibilidad inicial entre aceite y metanol
    \item \textbf{Desactivación del catalizador:} Posible lixiviación o envenenamiento de CaO
    \item \textbf{Acumulación de glicerol:} Subproducto que puede inhibir la reacción
\end{enumerate}

\subsection{Reproducibilidad del Método}

La reproducibilidad perfecta observada entre los Experimentos 1 y 3 (CV = 0.95\%) es excepcional y plantea dos escenarios:

\textbf{Escenario 1 - Reproducibilidad real:}
Si los datos son genuinamente independientes, esto representa un nivel de control experimental extraordinario, comparable con procesos industriales optimizados. Factores que podrían contribuir:
\begin{itemize}
    \item Control preciso de temperatura y agitación
    \item Preparación consistente de catalizador
    \item Calidad uniforme del aceite usado
    \item Metodología cromatográfica estandarizada
\end{itemize}

\textbf{Escenario 2 - Datos duplicados:}
Los valores idénticos también podrían indicar que los archivos del Experimento 3 son copias del Experimento 1, lo cual requiere verificación con los cuadernos de laboratorio originales.

\subsection{Discrepancia entre Conversión y Pureza}

La observación más crítica de este estudio es la gran discrepancia entre la conversión aparente (98-99\%) y la pureza de FAMEs (35-40\%). Esta inconsistencia sugiere varios problemas metodológicos potenciales:

\subsubsection{Problemas de Identificación de Picos}

\begin{enumerate}
    \item \textbf{Asignación incorrecta de rangos de TR:} Los rangos de tiempo de retención utilizados para clasificar FAMEs (6.5-11.5 min) pueden incluir compuestos no-FAMEs o excluir algunos FAMEs importantes.

    \item \textbf{Sobreposición de picos:} Monoglicéridos, diglicéridos y algunos FAMEs pueden tener tiempos de retención similares, causando cuantificación errónea.

    \item \textbf{Compuestos no identificados:} Picos grandes no asignados a categorías específicas.
\end{enumerate}

\subsubsection{Problemas de Integración}

El alto contenido reportado de monoglicéridos (86\%) y diglicéridos (72\%) que suma más del 100\% indica:
\begin{itemize}
    \item Error en la normalización de áreas
    \item Sobreposición de rangos de tiempo de retención entre categorías
    \item Necesidad de revisión de los rangos definidos para cada componente
\end{itemize}

\subsubsection{Recomendaciones para Corrección}

\begin{enumerate}
    \item \textbf{Análisis con patrones certificados:} Utilizar estándares de FAMEs individuales (C16:0, C18:0, C18:1, C18:2, etc.) para identificar correctamente tiempos de retención específicos.

    \item \textbf{Revisión de método cromatográfico:} Verificar programa de temperatura, tipo de columna y condiciones que permitan mejor separación de picos.

    \item \textbf{Integración manual:} Revisar la integración automática de picos, especialmente en regiones críticas.

    \item \textbf{Método normalizado:} Implementar métodos estándar como EN 14103 o ASTM D6584 específicos para biodiesel.
\end{enumerate}

\subsection{Efecto del Catalizador CaO}

El óxido de calcio demostró ser efectivo para la transesterificación, con conversiones superiores al 96\% en todas las condiciones ensayadas. Ventajas observadas:

\begin{itemize}
    \item Alta actividad catalítica a temperaturas moderadas (50-55°C)
    \item Tiempo de reacción relativamente corto ($<$2 horas)
    \item Catalizador heterogéneo (fácil separación)
\end{itemize}

Sin embargo, es necesario evaluar:
\begin{itemize}
    \item Reutilización del catalizador
    \item Contenido de calcio residual en el biodiesel
    \item Efecto de impurezas del aceite usado en la actividad catalítica
\end{itemize}

\subsection{Comparación con Normas de Calidad}

Según la norma EN 14214 para biodiesel, los requisitos principales son:

\begin{table}[H]
\centering
\caption{Comparación con norma EN 14214}
\begin{tabular}{@{}lcc@{}}
\toprule
\textbf{Parámetro} & \textbf{EN 14214 Límite} & \textbf{Este Estudio} \\
\midrule
Contenido de ésteres (\%) & $\geq$ 96.5 & 35-52* \\
Monoglicéridos (\%) & $\leq$ 0.80 & 83-89* \\
Diglicéridos (\%) & $\leq$ 0.20 & 63-76* \\
Triglicéridos (\%) & $\leq$ 0.20 & 6-8* \\
\bottomrule
\end{tabular}
\end{table}

*Nota: Valores requieren verificación debido a inconsistencias metodológicas detectadas.

Si los valores de conversión (98-99\%) son correctos, el biodiesel producido cumpliría con el requisito de contenido mínimo de ésteres. Sin embargo, los altos contenidos de glicéridos reportados exceden significativamente los límites permitidos, lo cual contradiría una conversión del 98\%.

\subsection{Limitaciones del Estudio}

Este análisis identificó varias limitaciones importantes:

\begin{enumerate}
    \item \textbf{Identificación de compuestos:} Basada únicamente en rangos de tiempo de retención sin confirmación con patrones
    \item \textbf{Cuantificación relativa:} No se realizó cuantificación absoluta excepto con heptano
    \item \textbf{Falta de validación de método:} No se reportan parámetros como linealidad, precisión, exactitud
    \item \textbf{Datos incompletos:} Falta información sobre propiedades físicas (viscosidad, densidad, índice de acidez)
    \item \textbf{Análisis de glicerol:} No se cuantificó glicerol libre y total
\end{enumerate}

\subsection{Recomendaciones para Futuros Experimentos}

Para mejorar la calidad y confiabilidad de futuros análisis:

\subsubsection{Mejoras Metodológicas}

\begin{enumerate}
    \item Implementar método estandarizado (EN 14103 o ASTM D6584)
    \item Utilizar patrones certificados de FAMEs y glicéridos
    \item Realizar análisis con GC-MS para identificación inequívoca
    \item Incluir curvas de calibración para cuantificación absoluta
    \item Realizar triplicados de cada muestra
\end{enumerate}

\subsubsection{Análisis Complementarios}

\begin{enumerate}
    \item Determinación de propiedades físicas (viscosidad cinemática, densidad)
    \item Análisis de índice de acidez y contenido de agua
    \item Cuantificación de glicerol libre y total
    \item Análisis de contenido de metanol residual
    \item Caracterización del perfil de ácidos grasos
\end{enumerate}

\subsubsection{Optimización del Proceso}

\begin{enumerate}
    \item Evaluar diferentes porcentajes de catalizador (0.5-2\%)
    \item Estudiar efecto de temperatura (40-65°C)
    \item Optimizar relación molar metanol:aceite (3:1 a 9:1)
    \item Investigar tiempo óptimo de reacción
    \item Evaluar reutilización del catalizador CaO
\end{enumerate}

\section{Conclusiones}

\begin{enumerate}
    \item Se analizaron exitosamente 24 muestras de biodiesel de 4 experimentos independientes mediante cromatografía de gases.

    \item La conversión aparente de triglicéridos a FAMEs fue consistentemente alta (96.8-99.6\%), indicando que el proceso de transesterificación con CaO fue efectivo.

    \item El catalizador CaO demostró alta actividad, alcanzando conversiones superiores al 98\% en menos de 48 minutos a 50-55°C.

    \item La reproducibilidad del método fue excelente, con coeficiente de variación menor al 1\%.

    \item Se identificó una discrepancia crítica entre la conversión calculada (98\%) y la pureza de FAMEs (35-40\%), que requiere investigación adicional.

    \item Los altos contenidos reportados de monoglicéridos (86\%) y diglicéridos (72\%) son inconsistentes con las conversiones observadas, sugiriendo problemas en la asignación de picos o integración.

    \item Es necesario implementar métodos estandarizados (EN 14103 o ASTM D6584) con patrones certificados para validar los resultados.

    \item El proceso de transesterificación estudiado muestra potencial para producción de biodiesel, pero requiere optimización y caracterización más completa para confirmar cumplimiento con normas internacionales.
\end{enumerate}
