\documentclass[12pt,a4paper]{article}
\usepackage[utf8]{inputenc}
\usepackage[spanish]{babel}
\usepackage{amsmath}
\usepackage{graphicx}
\usepackage{booktabs}
\usepackage{longtable}
\usepackage{geometry}
\usepackage{hyperref}
\usepackage{float}
\usepackage{listings}
\usepackage{xcolor}
\usepackage{caption}
\usepackage{subcaption}

\geometry{margin=2.5cm}

\title{Análisis de Cromatogramas de Biodiesel:\\
Producción por Transesterificación de Aceite Usado de Cocina}
\author{Análisis de Datos de Experimentos MORAN}
\date{Noviembre 2025}

\begin{document}

\maketitle

\begin{abstract}
Este documento presenta un análisis sistemático de datos cromatográficos obtenidos durante la producción de biodiesel mediante transesterificación de aceite usado de cocina con catalizador CaO. Se analizaron 18 muestras distribuidas en 3 experimentos realizados entre octubre y noviembre de 2025. El análisis incluye la caracterización de la conversión de triglicéridos a ésteres metílicos de ácidos grasos (FAMEs), evaluación de la calidad del biodiesel producido y comparación entre experimentos. Se implementó un sistema de nomenclatura alfanumérica (E1a, E1b, E2a, etc.) para facilitar la identificación y trazabilidad de las muestras.
\end{abstract}

\tableofcontents
\newpage

\section{Nomenclatura y Sistema de Identificación}

\subsection{Nomenclatura Actualizada}

Para facilitar la identificación y trazabilidad de las muestras, se implementó un sistema de nomenclatura alfanumérica consistente:

\begin{itemize}
    \item \textbf{Formato:} E[número\_experimento][letra\_muestra]
    \item \textbf{Ejemplos:}
    \begin{itemize}
        \item E1a: Experimento 1, primera muestra (cronológicamente)
        \item E1b: Experimento 1, segunda muestra
        \item E2c: Experimento 2, tercera muestra
    \end{itemize}
    \item \textbf{Ventajas:}
    \begin{itemize}
        \item Identificación rápida del experimento de origen
        \item Orden cronológico explícito dentro de cada experimento
        \item Nomenclatura compacta y fácil de usar en gráficas
        \item Consistencia entre archivos de datos, scripts y documentación
    \end{itemize}
\end{itemize}

\subsection{Tabla de Mapeo}

La Tabla~\ref{tab:mapeo} muestra la correspondencia entre la nomenclatura original de los archivos y la nomenclatura actualizada:

\begin{table}[H]
\centering
\caption{Mapeo de nomenclatura original a actualizada}
\label{tab:mapeo}
\begin{tabular}{llll}
\toprule
\textbf{Experimento} & \textbf{Nombre Original} & \textbf{Nomenclatura} & \textbf{Orden} \\
\midrule
Experimento 1 & 2.1 & E1a & 1 \\
Experimento 1 & 3.1 & E1b & 2 \\
Experimento 1 & 5.1 & E1c & 3 \\
Experimento 1 & 6.1 & E1d & 4 \\
Experimento 1 & 9.1 & E1e & 5 \\
Experimento 1 & 12.1 & E1f & 6 \\
\midrule
Experimento 2 & 1.1 & E2a & 1 \\
Experimento 2 & 8.1 & E2b & 2 \\
Experimento 2 & 10.1 & E2c & 3 \\
Experimento 2 & 11.1 & E2d & 4 \\
Experimento 2 & SN1 & E2e & 5 \\
Experimento 2 & SN2 & E2f & 6 \\
\midrule
Experimento 3 & RXN5 & E3a & 1 \\
Experimento 3 & RXN10 & E3b & 2 \\
Experimento 3 & MITAD & E3c & 3 \\
Experimento 3 & FINAL & E3d & 4 \\
Experimento 3 & 6.2 & E3e & 5 \\
Experimento 3 & 12.2 & E3f & 6 \\
\bottomrule
\end{tabular}
\end{table}

\section{Origen de los Datos}

\subsection{Descripción General}

Los datos analizados provienen de tres experimentos independientes de producción de biodiesel mediante transesterificación catalítica heterogénea. Todos los análisis fueron realizados mediante Cromatografía de Gases (GC) con detector FID, utilizando heptano como estándar interno.

\textbf{Nota importante:} Un cuarto experimento realizado el 24/10/2025 fue identificado como duplicado exacto del Experimento 1 mediante análisis de checksums MD5 y comparación numérica, por lo que fue excluido del análisis.

\subsection{Experimentos Analizados}

\subsubsection{Experimento 1 (03/10/2025)}

\begin{itemize}
    \item \textbf{Archivo fuente:} \texttt{Experimento1/Cromatograma/cromatogramaExperimento1.xlsx}
    \item \textbf{Tipo:} Reacción de transesterificación con monitoreo temporal
    \item \textbf{Condiciones:}
    \begin{itemize}
        \item Aceite usado de cocina: 100 mL (91.53 g)
        \item Metanol: 25.51 mL
        \item Catalizador: CaO 1\% p/p
        \item Temperatura: 50-55°C
        \item Agitación: 100-600 RPM
        \item Duración: 120 minutos
        \item Relación molar: 6:1
    \end{itemize}
    \item \textbf{Muestras:} 6 (E1a--E1f)
    \item \textbf{Frecuencia de muestreo:} Cada 24 minutos
\end{itemize}

\subsubsection{Experimento 2 (20/10/2025)}

\begin{itemize}
    \item \textbf{Archivo fuente:} \texttt{20251020\_MORAN 20-10-25/2025-10-20 MORAN.XLS}
    \item \textbf{Tipo:} Análisis de diferentes condiciones experimentales
    \item \textbf{Muestras:} 6 (E2a--E2f)
    \item \textbf{Nota:} Incluye muestras con diferentes nomenclaturas originales
\end{itemize}

\subsubsection{Experimento 3 (07/11/2025)}

\begin{itemize}
    \item \textbf{Archivo fuente:} \texttt{20251107\_MORAN 7-11-25/2025-11-07 MORAN.XLS}
    \item \textbf{Tipo:} Nuevas reacciones con puntos de control temporales
    \item \textbf{Muestras:} 6 (E3a--E3f)
    \item \textbf{Nota:} Incluye muestras RXN (reacciones a diferentes tiempos), puntos MITAD/FINAL, y repeticiones
\end{itemize}

\section{Metodología de Análisis}

\subsection{Identificación de Componentes}

Los componentes se identificaron mediante rangos de tiempo de retención (TR):

\begin{table}[H]
\centering
\caption{Rangos de tiempo de retención para identificación de componentes}
\begin{tabular}{lcc}
\toprule
\textbf{Componente} & \textbf{TR mínimo (min)} & \textbf{TR máximo (min)} \\
\midrule
Heptano (EI) & 0.96 & 0.99 \\
Metanol residual & 2.20 & 2.35 \\
FAMEs (Biodiesel) & 6.50 & 11.50 \\
Monoglicéridos (MAG) & 7.40 & 8.60 \\
Diglicéridos (DAG) & 7.70 & 8.40 \\
Triglicéridos (TAG) & 7.00 & 7.25 \\
\bottomrule
\end{tabular}
\end{table}

\subsection{Cálculo de Parámetros de Calidad}

\subsubsection{Conversión a FAMEs}

La conversión representa el porcentaje de productos (FAMEs) respecto al total de compuestos, excluyendo el estándar interno:

\begin{equation}
\text{Conversión (\%)} = \frac{A_{\text{FAMEs}}}{A_{\text{total}} - A_{\text{heptano}}} \times 100
\end{equation}

\subsubsection{Pureza del Biodiesel}

La pureza indica el porcentaje de FAMEs respecto a la suma de todos los productos de reacción:

\begin{equation}
\text{Pureza (\%)} = \frac{A_{\text{FAMEs}}}{A_{\text{FAMEs}} + A_{\text{MAG}} + A_{\text{DAG}} + A_{\text{TAG}}} \times 100
\end{equation}

\subsubsection{Contenido de Glicéridos}

El contenido de cada tipo de glicérido se calcula como:

\begin{equation}
\text{Glicéridos (\%)} = \frac{A_{\text{glicérido}}}{A_{\text{total}} - A_{\text{heptano}}} \times 100
\end{equation}

\section{Resultados y Discusión}

\subsection{Evolución Temporal del Experimento 1}

El Experimento 1 proporciona información detallada sobre la cinética de la reacción de transesterificación. La Figura~\ref{fig:evolucion_exp1} muestra la evolución de la conversión a biodiesel y otros parámetros de calidad en función del tiempo.

\begin{figure}[H]
\centering
\includegraphics[width=0.95\textwidth]{Procesados/figuras/fig1_evolucion_temporal_exp1.png}
\caption{Evolución temporal de la conversión a biodiesel y parámetros de calidad para el Experimento 1. Las muestras están ordenadas cronológicamente (E1a--E1f), tomadas cada 24 minutos.}
\label{fig:evolucion_exp1}
\end{figure}

\textbf{Observaciones clave:}
\begin{itemize}
    \item La conversión alcanza un 96.83\% a tiempo cero (E1a) y se incrementa rápidamente hasta 99.29\% a los 48 minutos (E1c)
    \item Se observa una estabilización en torno a 99.2--99.3\% entre 48 y 120 minutos
    \item La pureza del biodiesel disminuye ligeramente con el tiempo debido al incremento en glicéridos residuales
    \item El ordenamiento cronológico correcto de las muestras es crítico para interpretar correctamente la cinética de reacción
\end{itemize}

\subsection{Comparación entre Experimentos}

La Figura~\ref{fig:comparacion} muestra la comparación de conversión entre todas las muestras de los tres experimentos.

\begin{figure}[H]
\centering
\includegraphics[width=0.95\textwidth]{Procesados/figuras/fig2_comparacion_experimentos.png}
\caption{Comparación de conversión a biodiesel entre todos los experimentos. Las líneas verticales separan los tres experimentos.}
\label{fig:comparacion}
\end{figure}

\subsection{Composición de las Muestras}

La Figura~\ref{fig:composicion} presenta un análisis de la composición relativa de FAMEs y glicéridos residuales en todas las muestras.

\begin{figure}[H]
\centering
\includegraphics[width=0.95\textwidth]{Procesados/figuras/fig3_composicion_apilada.png}
\caption{Composición relativa de FAMEs y glicéridos residuales para todas las muestras analizadas.}
\label{fig:composicion}
\end{figure}

\subsection{Comparación Temporal entre Experimentos}

La Figura~\ref{fig:temporal} compara los promedios de conversión y pureza entre los tres experimentos en orden cronológico.

\begin{figure}[H]
\centering
\includegraphics[width=0.95\textwidth]{Procesados/figuras/fig4_comparacion_temporal.png}
\caption{Evolución temporal de conversión y pureza promedio entre experimentos. El Experimento 2 muestra la mayor conversión promedio (99.32\%).}
\label{fig:temporal}
\end{figure}

\subsection{Análisis Estadístico}

La Figura~\ref{fig:estadisticas} presenta boxplots e histogramas de los principales parámetros de calidad.

\begin{figure}[H]
\centering
\includegraphics[width=0.95\textwidth]{Procesados/figuras/fig5_estadisticas_boxplot.png}
\caption{Distribuciones estadísticas de conversión y pureza. Los boxplots muestran la variabilidad dentro de cada experimento, mientras que los histogramas muestran la distribución global.}
\label{fig:estadisticas}
\end{figure}

\subsection{Relación Conversión--Pureza}

La Figura~\ref{fig:scatter} explora la relación entre conversión y pureza del biodiesel.

\begin{figure}[H]
\centering
\includegraphics[width=0.85\textwidth]{Procesados/figuras/fig6_scatter_conversion_pureza.png}
\caption{Scatter plot mostrando la relación entre conversión y pureza. Se observa una tendencia inversa: mayor conversión correlaciona con menor pureza, sugiriendo acumulación de glicéridos.}
\label{fig:scatter}
\end{figure}

\subsection{Contenido de Glicéridos Residuales}

La Figura~\ref{fig:gliceridos} compara el contenido promedio de glicéridos residuales entre experimentos.

\begin{figure}[H]
\centering
\includegraphics[width=0.90\textwidth]{Procesados/figuras/fig7_gliceridos_promedio.png}
\caption{Contenido promedio de monoglicéridos, diglicéridos y triglicéridos por experimento. Los valores elevados sugieren posibles solapamientos en los rangos de TR utilizados.}
\label{fig:gliceridos}
\end{figure}

\subsection{Área de Picos FAMEs}

La Figura~\ref{fig:area_fames} muestra el área total de picos FAMEs por muestra.

\begin{figure}[H]
\centering
\includegraphics[width=0.95\textwidth]{Procesados/figuras/fig8_area_fames.png}
\caption{Área total de picos FAMEs por muestra. Las variaciones reflejan diferencias en la concentración de biodiesel y en la respuesta del detector.}
\label{fig:area_fames}
\end{figure}

\subsection{Número de Picos FAMEs Identificados}

La Figura~\ref{fig:picos} presenta el número de picos FAMEs identificados en cada muestra.

\begin{figure}[H]
\centering
\includegraphics[width=0.95\textwidth]{Procesados/figuras/fig9_picos_fames.png}
\caption{Número de picos FAMEs identificados por muestra. La variación (30--43 picos) refleja diferencias en la composición de ácidos grasos del aceite usado.}
\label{fig:picos}
\end{figure}

\subsection{Mapa de Calor de Parámetros de Calidad}

La Figura~\ref{fig:heatmap} presenta un heatmap con todos los parámetros de calidad normalizados.

\begin{figure}[H]
\centering
\includegraphics[width=0.95\textwidth]{Procesados/figuras/fig10_heatmap_calidad.png}
\caption{Mapa de calor de parámetros de calidad. Los valores están normalizados (0--1) para visualización, con los valores reales mostrados en cada celda.}
\label{fig:heatmap}
\end{figure}

\section{Conclusiones}

\begin{enumerate}
    \item Se implementó exitosamente un sistema de nomenclatura alfanumérica (E1a--E3f) que facilita la identificación y trazabilidad de las 18 muestras analizadas.

    \item La conversión a biodiesel es consistentemente alta en todos los experimentos (96--99.5\%), demostrando la efectividad del proceso de transesterificación con CaO como catalizador.

    \item Se identificó y removió un experimento duplicado (24/10/2025) mediante análisis de checksums MD5, asegurando la integridad del análisis.

    \item El Experimento 2 muestra la mayor conversión promedio (99.32\% ± 0.17\%) con la menor variabilidad (CV = 0.17\%).

    \item Los valores elevados de glicéridos (>100\% al sumar MAG+DAG+TAG) sugieren solapamiento en los rangos de tiempo de retención utilizados, requiriendo optimización del método cromatográfico.

    \item La relación inversa observada entre conversión y pureza indica que el incremento en productos de reacción incluye tanto FAMEs como glicéridos intermedios.

    \item El pipeline de análisis desarrollado (extracción, procesamiento, visualización) es robusto y reproducible, facilitando análisis futuros de experimentos similares.
\end{enumerate}

\section{Recomendaciones}

\begin{enumerate}
    \item Optimizar los rangos de tiempo de retención para glicéridos para evitar solapamientos.

    \item Implementar curvas de calibración con estándares certificados de MAG, DAG y TAG.

    \item Validar el método de cuantificación mediante comparación con técnicas complementarias (GC-MS, HPLC).

    \item Extender el análisis temporal más allá de 120 minutos para caracterizar completamente la cinética de reacción.

    \item Evaluar el efecto de la desactivación del catalizador CaO mediante estudios de reutilización.
\end{enumerate}

\end{document}
